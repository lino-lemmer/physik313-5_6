% Copyright © 2013 Martin Ueding <dev@martin-ueding.de>

% Copyright © 2012-2013 Martin Ueding <dev@martin-ueding.de>

% This is my general purpose LaTeX header file for writing German documents.
% Ideally, you include this using a simple ``% Copyright © 2012-2013 Martin Ueding <dev@martin-ueding.de>

% This is my general purpose LaTeX header file for writing German documents.
% Ideally, you include this using a simple ``% Copyright © 2012-2013 Martin Ueding <dev@martin-ueding.de>

% This is my general purpose LaTeX header file for writing German documents.
% Ideally, you include this using a simple ``\input{header.tex}`` in your main
% document and start with ``\title`` and ``\begin{document}`` afterwards.

% If you need to add additional packages, I recommend not doing this in this
% file, but in your main document. That way, you can just drop in a new
% ``header.tex`` and get all the new commands without having to merge manually.

% Since this file encorporates a CC-BY-SA fragment, this whole files is
% licensed under the CC-BY-SA license.

\documentclass[11pt, ngerman, fleqn, DIV=15, headinclude, BCOR=2cm]{scrartcl}

\usepackage{graphicx}

% Environment to quote the problem. Currently, this is just a new name for the
% quote environment.
\newenvironment{problem}{\begin{quote}\textsf{\textbf{Aufgabenstellung:}}\quad}{\end{quote}}

\setkomafont{caption}{\sf}
\setkomafont{captionlabel}{\usekomafont{caption}}

%%%%%%%%%%%%%%%%%%%%%%%%%%%%%%%%%%%%%%%%%%%%%%%%%%%%%%%%%%%%%%%%%%%%%%%%%%%%%%%
%                                Locale, date                                 %
%%%%%%%%%%%%%%%%%%%%%%%%%%%%%%%%%%%%%%%%%%%%%%%%%%%%%%%%%%%%%%%%%%%%%%%%%%%%%%%

\usepackage{babel}
\usepackage[iso]{isodate}

%%%%%%%%%%%%%%%%%%%%%%%%%%%%%%%%%%%%%%%%%%%%%%%%%%%%%%%%%%%%%%%%%%%%%%%%%%%%%%%
%                          Margins and other spacing                          %
%%%%%%%%%%%%%%%%%%%%%%%%%%%%%%%%%%%%%%%%%%%%%%%%%%%%%%%%%%%%%%%%%%%%%%%%%%%%%%%

\usepackage[parfill]{parskip}
\usepackage{setspace}
\usepackage[activate]{microtype}

\setlength{\columnsep}{2cm}

%%%%%%%%%%%%%%%%%%%%%%%%%%%%%%%%%%%%%%%%%%%%%%%%%%%%%%%%%%%%%%%%%%%%%%%%%%%%%%%
%                                    Color                                    %
%%%%%%%%%%%%%%%%%%%%%%%%%%%%%%%%%%%%%%%%%%%%%%%%%%%%%%%%%%%%%%%%%%%%%%%%%%%%%%%

\usepackage[usenames, dvipsnames]{xcolor}

\colorlet{darkred}{red!70!black}
\colorlet{darkblue}{blue!70!black}
\colorlet{darkgreen}{green!40!black}

%%%%%%%%%%%%%%%%%%%%%%%%%%%%%%%%%%%%%%%%%%%%%%%%%%%%%%%%%%%%%%%%%%%%%%%%%%%%%%%
%                         Font and font like settings                         %
%%%%%%%%%%%%%%%%%%%%%%%%%%%%%%%%%%%%%%%%%%%%%%%%%%%%%%%%%%%%%%%%%%%%%%%%%%%%%%%

% This replaces all fonts with Bitstream Charter, Bitstream Vera Sans and
% Bitstream Vera Mono. Math will be rendered in Charter.
\usepackage[charter, greekuppercase=italicized]{mathdesign}
\usepackage{beramono}
\usepackage{berasans}

% Bold, sans-serif tensors. This fragment is taken from “egreg” from
% http://tex.stackexchange.com/a/82747/8945 and licensed under `CC-BY-SA
% <https://creativecommons.org/licenses/by-sa/3.0/>`_.
\usepackage{bm}
\DeclareMathAlphabet{\mathsfit}{\encodingdefault}{\sfdefault}{m}{sl}
\SetMathAlphabet{\mathsfit}{bold}{\encodingdefault}{\sfdefault}{bx}{sl}
\newcommand{\tens}[1]{\bm{\mathsfit{#1}}}

% Bold vectors.
\renewcommand{\vec}[1]{\boldsymbol{#1}}

%%%%%%%%%%%%%%%%%%%%%%%%%%%%%%%%%%%%%%%%%%%%%%%%%%%%%%%%%%%%%%%%%%%%%%%%%%%%%%%
%                               Input encoding                                %
%%%%%%%%%%%%%%%%%%%%%%%%%%%%%%%%%%%%%%%%%%%%%%%%%%%%%%%%%%%%%%%%%%%%%%%%%%%%%%%

\usepackage[T1]{fontenc}
\usepackage[utf8]{inputenc}

%%%%%%%%%%%%%%%%%%%%%%%%%%%%%%%%%%%%%%%%%%%%%%%%%%%%%%%%%%%%%%%%%%%%%%%%%%%%%%%
%                         Hyperrefs and PDF metadata                          %
%%%%%%%%%%%%%%%%%%%%%%%%%%%%%%%%%%%%%%%%%%%%%%%%%%%%%%%%%%%%%%%%%%%%%%%%%%%%%%%

\usepackage{hyperref}
\usepackage{lastpage}

% This sets the author in the properties of the PDF as well. If you want to
% change it, just override it with another ``\hypersetup`` call.
\hypersetup{
	breaklinks=false,
	citecolor=darkgreen,
	colorlinks=true,
	linkcolor=darkblue,
	menucolor=black,
	pdfauthor={Martin Ueding},
	urlcolor=darkblue,
}

%%%%%%%%%%%%%%%%%%%%%%%%%%%%%%%%%%%%%%%%%%%%%%%%%%%%%%%%%%%%%%%%%%%%%%%%%%%%%%%
%                               Math Operators                                %
%%%%%%%%%%%%%%%%%%%%%%%%%%%%%%%%%%%%%%%%%%%%%%%%%%%%%%%%%%%%%%%%%%%%%%%%%%%%%%%

% AMS environments like ``align`` and theorems like ``proof``.
\usepackage{amsmath}
\usepackage{amsthm}

% Common math constructs like partial derivatives.
\usepackage{commath}

% Physical units.
\usepackage[output-decimal-marker={,}]{siunitx}

% Since I use mathdesign with italic uppercase greek characters, the Ohm unit will be displayed with an italic Ω by default. Units have to be roman, so this forces it the right way.
\DeclareSIUnit{\ohm}{$\Omegaup$}
\DeclareSIUnit{\division}{DIV}
\DeclareSIUnit{\voltss}{$\mathrm{V_{SS}}$}

% Word like operators.
\DeclareMathOperator{\acosh}{arcosh}
\DeclareMathOperator{\arcosh}{arcosh}
\DeclareMathOperator{\arcsinh}{arsinh}
\DeclareMathOperator{\arsinh}{arsinh}
\DeclareMathOperator{\asinh}{arsinh}
\DeclareMathOperator{\card}{card}
\DeclareMathOperator{\csch}{cshs}
\DeclareMathOperator{\diam}{diam}
\DeclareMathOperator{\sech}{sech}
\renewcommand{\Im}{\mathop{{}\mathrm{Im}}\nolimits}
\renewcommand{\Re}{\mathop{{}\mathrm{Re}}\nolimits}

% Fourier transform.
\DeclareMathOperator{\fourier}{\ensuremath{\mathcal{F}}}

% Roman versions of “e” and “i” to serve as Euler's number and the imaginary
% constant.
\newcommand{\ee}{\eup}
\newcommand{\eup}{\mathrm e}
\newcommand{\ii}{\iup}
\newcommand{\iup}{\mathrm i}

% Symbols for the various mathematical fields (natural numbers, integers,
% rational numbers, real numbers, complex numbers).
\newcommand{\C}{\ensuremath{\mathbb C}}
\newcommand{\N}{\ensuremath{\mathbb N}}
\newcommand{\Q}{\ensuremath{\mathbb Q}}
\newcommand{\R}{\ensuremath{\mathbb R}}
\newcommand{\Z}{\ensuremath{\mathbb Z}}

% Shape like operators.
\DeclareMathOperator{\dalambert}{\Box}
\DeclareMathOperator{\laplace}{\bigtriangleup}
\newcommand{\curl}{\vnabla \times}
\newcommand{\divergence}[1]{\inner{\vnabla}{#1}}
\newcommand{\vnabla}{\vec \nabla}

\newcommand{\half}{\frac 12}

% Unit vector (German „Einheitsvektor“).
\newcommand{\ev}{\hat{\vec e}}

% Scientific notation for large numbers.
\newcommand{\e}[1]{\cdot 10^{#1}}

% Mathematician's notation for the inner (scalar, dot) product.
\newcommand{\bracket}[1]{\left\langle #1 \right\rangle}
\newcommand{\inner}[2]{\bracket{#1, #2}}

% Placeholders.
\newcommand{\emesswert}{\del{\messwert \pm \messwert}}
\newcommand{\fehlt}{\textcolor{darkred}{Hier fehlen noch Inhalte.}}
\newcommand{\messwert}{\textcolor{blue}{\square}}
\newcommand{\punkte}{\phantom{xxxxx}}
\newcommand{\punktevon}[1]{\begin{flushright}/ #1\end{flushright}}

% Separator for equations on a single line.
\newcommand{\eqnsep}{,\quad}

% Quantum Mechanics
\usepackage{braket}

%%%%%%%%%%%%%%%%%%%%%%%%%%%%%%%%%%%%%%%%%%%%%%%%%%%%%%%%%%%%%%%%%%%%%%%%%%%%%%%
%                                  Headings                                   %
%%%%%%%%%%%%%%%%%%%%%%%%%%%%%%%%%%%%%%%%%%%%%%%%%%%%%%%%%%%%%%%%%%%%%%%%%%%%%%%

% This will set fancy headings to the top of the page. The page number will be
% accompanied by the total number of pages. That way, you will know if any page
% is missing.
%
% If you do not want this for your document, you can just use
% ``\pagestyle{plain}``.

\usepackage{scrpage2}

\pagestyle{scrheadings}
\automark{section}
\cfoot{\footnotesize{Seite \thepage\ / \pageref{LastPage}}}
\chead{}
\ihead{}
\ohead{\rightmark}
\setheadsepline{.4pt}

%%%%%%%%%%%%%%%%%%%%%%%%%%%%%%%%%%%%%%%%%%%%%%%%%%%%%%%%%%%%%%%%%%%%%%%%%%%%%%%
%                            Bibliography (BibTeX)                            %
%%%%%%%%%%%%%%%%%%%%%%%%%%%%%%%%%%%%%%%%%%%%%%%%%%%%%%%%%%%%%%%%%%%%%%%%%%%%%%%

\newcommand{\bibliographyfile}{../../../zentrale_BibTeX/Central}
\bibliographystyle{apalike2}

%%%%%%%%%%%%%%%%%%%%%%%%%%%%%%%%%%%%%%%%%%%%%%%%%%%%%%%%%%%%%%%%%%%%%%%%%%%%%%%
%                                Abbreviations                                %
%%%%%%%%%%%%%%%%%%%%%%%%%%%%%%%%%%%%%%%%%%%%%%%%%%%%%%%%%%%%%%%%%%%%%%%%%%%%%%%

\newcommand{\dhabk}{\mbox{d.\,h.}}

%%%%%%%%%%%%%%%%%%%%%%%%%%%%%%%%%%%%%%%%%%%%%%%%%%%%%%%%%%%%%%%%%%%%%%%%%%%%%%%
%                                  Licences                                   %
%%%%%%%%%%%%%%%%%%%%%%%%%%%%%%%%%%%%%%%%%%%%%%%%%%%%%%%%%%%%%%%%%%%%%%%%%%%%%%%

\usepackage{ccicons}

\newcommand{\ccbysadetext}{%
	\begin{small}
		Dieses Werk bzw. Inhalt steht unter einer
		\href{http://creativecommons.org/licenses/by-sa/3.0/deed.de}{%
			Creative Commons Namensnennung - Weitergabe unter gleichen
		Bedingungen 3.0 Unported Lizenz}.
	\end{small}
}

\newcommand{\ccbysadetitle}{%
	Lizenz: \href{http://creativecommons.org/licenses/by-sa/3.0/deed.de}
	{CC-BY-SA 3.0 \ccbysa}
}
`` in your main
% document and start with ``\title`` and ``\begin{document}`` afterwards.

% If you need to add additional packages, I recommend not doing this in this
% file, but in your main document. That way, you can just drop in a new
% ``header.tex`` and get all the new commands without having to merge manually.

% Since this file encorporates a CC-BY-SA fragment, this whole files is
% licensed under the CC-BY-SA license.

\documentclass[11pt, ngerman, fleqn, DIV=15, headinclude, BCOR=2cm]{scrartcl}

\usepackage{graphicx}

% Environment to quote the problem. Currently, this is just a new name for the
% quote environment.
\newenvironment{problem}{\begin{quote}\textsf{\textbf{Aufgabenstellung:}}\quad}{\end{quote}}

\setkomafont{caption}{\sf}
\setkomafont{captionlabel}{\usekomafont{caption}}

%%%%%%%%%%%%%%%%%%%%%%%%%%%%%%%%%%%%%%%%%%%%%%%%%%%%%%%%%%%%%%%%%%%%%%%%%%%%%%%
%                                Locale, date                                 %
%%%%%%%%%%%%%%%%%%%%%%%%%%%%%%%%%%%%%%%%%%%%%%%%%%%%%%%%%%%%%%%%%%%%%%%%%%%%%%%

\usepackage{babel}
\usepackage[iso]{isodate}

%%%%%%%%%%%%%%%%%%%%%%%%%%%%%%%%%%%%%%%%%%%%%%%%%%%%%%%%%%%%%%%%%%%%%%%%%%%%%%%
%                          Margins and other spacing                          %
%%%%%%%%%%%%%%%%%%%%%%%%%%%%%%%%%%%%%%%%%%%%%%%%%%%%%%%%%%%%%%%%%%%%%%%%%%%%%%%

\usepackage[parfill]{parskip}
\usepackage{setspace}
\usepackage[activate]{microtype}

\setlength{\columnsep}{2cm}

%%%%%%%%%%%%%%%%%%%%%%%%%%%%%%%%%%%%%%%%%%%%%%%%%%%%%%%%%%%%%%%%%%%%%%%%%%%%%%%
%                                    Color                                    %
%%%%%%%%%%%%%%%%%%%%%%%%%%%%%%%%%%%%%%%%%%%%%%%%%%%%%%%%%%%%%%%%%%%%%%%%%%%%%%%

\usepackage[usenames, dvipsnames]{xcolor}

\colorlet{darkred}{red!70!black}
\colorlet{darkblue}{blue!70!black}
\colorlet{darkgreen}{green!40!black}

%%%%%%%%%%%%%%%%%%%%%%%%%%%%%%%%%%%%%%%%%%%%%%%%%%%%%%%%%%%%%%%%%%%%%%%%%%%%%%%
%                         Font and font like settings                         %
%%%%%%%%%%%%%%%%%%%%%%%%%%%%%%%%%%%%%%%%%%%%%%%%%%%%%%%%%%%%%%%%%%%%%%%%%%%%%%%

% This replaces all fonts with Bitstream Charter, Bitstream Vera Sans and
% Bitstream Vera Mono. Math will be rendered in Charter.
\usepackage[charter, greekuppercase=italicized]{mathdesign}
\usepackage{beramono}
\usepackage{berasans}

% Bold, sans-serif tensors. This fragment is taken from “egreg” from
% http://tex.stackexchange.com/a/82747/8945 and licensed under `CC-BY-SA
% <https://creativecommons.org/licenses/by-sa/3.0/>`_.
\usepackage{bm}
\DeclareMathAlphabet{\mathsfit}{\encodingdefault}{\sfdefault}{m}{sl}
\SetMathAlphabet{\mathsfit}{bold}{\encodingdefault}{\sfdefault}{bx}{sl}
\newcommand{\tens}[1]{\bm{\mathsfit{#1}}}

% Bold vectors.
\renewcommand{\vec}[1]{\boldsymbol{#1}}

%%%%%%%%%%%%%%%%%%%%%%%%%%%%%%%%%%%%%%%%%%%%%%%%%%%%%%%%%%%%%%%%%%%%%%%%%%%%%%%
%                               Input encoding                                %
%%%%%%%%%%%%%%%%%%%%%%%%%%%%%%%%%%%%%%%%%%%%%%%%%%%%%%%%%%%%%%%%%%%%%%%%%%%%%%%

\usepackage[T1]{fontenc}
\usepackage[utf8]{inputenc}

%%%%%%%%%%%%%%%%%%%%%%%%%%%%%%%%%%%%%%%%%%%%%%%%%%%%%%%%%%%%%%%%%%%%%%%%%%%%%%%
%                         Hyperrefs and PDF metadata                          %
%%%%%%%%%%%%%%%%%%%%%%%%%%%%%%%%%%%%%%%%%%%%%%%%%%%%%%%%%%%%%%%%%%%%%%%%%%%%%%%

\usepackage{hyperref}
\usepackage{lastpage}

% This sets the author in the properties of the PDF as well. If you want to
% change it, just override it with another ``\hypersetup`` call.
\hypersetup{
	breaklinks=false,
	citecolor=darkgreen,
	colorlinks=true,
	linkcolor=darkblue,
	menucolor=black,
	pdfauthor={Martin Ueding},
	urlcolor=darkblue,
}

%%%%%%%%%%%%%%%%%%%%%%%%%%%%%%%%%%%%%%%%%%%%%%%%%%%%%%%%%%%%%%%%%%%%%%%%%%%%%%%
%                               Math Operators                                %
%%%%%%%%%%%%%%%%%%%%%%%%%%%%%%%%%%%%%%%%%%%%%%%%%%%%%%%%%%%%%%%%%%%%%%%%%%%%%%%

% AMS environments like ``align`` and theorems like ``proof``.
\usepackage{amsmath}
\usepackage{amsthm}

% Common math constructs like partial derivatives.
\usepackage{commath}

% Physical units.
\usepackage[output-decimal-marker={,}]{siunitx}

% Since I use mathdesign with italic uppercase greek characters, the Ohm unit will be displayed with an italic Ω by default. Units have to be roman, so this forces it the right way.
\DeclareSIUnit{\ohm}{$\Omegaup$}
\DeclareSIUnit{\division}{DIV}
\DeclareSIUnit{\voltss}{$\mathrm{V_{SS}}$}

% Word like operators.
\DeclareMathOperator{\acosh}{arcosh}
\DeclareMathOperator{\arcosh}{arcosh}
\DeclareMathOperator{\arcsinh}{arsinh}
\DeclareMathOperator{\arsinh}{arsinh}
\DeclareMathOperator{\asinh}{arsinh}
\DeclareMathOperator{\card}{card}
\DeclareMathOperator{\csch}{cshs}
\DeclareMathOperator{\diam}{diam}
\DeclareMathOperator{\sech}{sech}
\renewcommand{\Im}{\mathop{{}\mathrm{Im}}\nolimits}
\renewcommand{\Re}{\mathop{{}\mathrm{Re}}\nolimits}

% Fourier transform.
\DeclareMathOperator{\fourier}{\ensuremath{\mathcal{F}}}

% Roman versions of “e” and “i” to serve as Euler's number and the imaginary
% constant.
\newcommand{\ee}{\eup}
\newcommand{\eup}{\mathrm e}
\newcommand{\ii}{\iup}
\newcommand{\iup}{\mathrm i}

% Symbols for the various mathematical fields (natural numbers, integers,
% rational numbers, real numbers, complex numbers).
\newcommand{\C}{\ensuremath{\mathbb C}}
\newcommand{\N}{\ensuremath{\mathbb N}}
\newcommand{\Q}{\ensuremath{\mathbb Q}}
\newcommand{\R}{\ensuremath{\mathbb R}}
\newcommand{\Z}{\ensuremath{\mathbb Z}}

% Shape like operators.
\DeclareMathOperator{\dalambert}{\Box}
\DeclareMathOperator{\laplace}{\bigtriangleup}
\newcommand{\curl}{\vnabla \times}
\newcommand{\divergence}[1]{\inner{\vnabla}{#1}}
\newcommand{\vnabla}{\vec \nabla}

\newcommand{\half}{\frac 12}

% Unit vector (German „Einheitsvektor“).
\newcommand{\ev}{\hat{\vec e}}

% Scientific notation for large numbers.
\newcommand{\e}[1]{\cdot 10^{#1}}

% Mathematician's notation for the inner (scalar, dot) product.
\newcommand{\bracket}[1]{\left\langle #1 \right\rangle}
\newcommand{\inner}[2]{\bracket{#1, #2}}

% Placeholders.
\newcommand{\emesswert}{\del{\messwert \pm \messwert}}
\newcommand{\fehlt}{\textcolor{darkred}{Hier fehlen noch Inhalte.}}
\newcommand{\messwert}{\textcolor{blue}{\square}}
\newcommand{\punkte}{\phantom{xxxxx}}
\newcommand{\punktevon}[1]{\begin{flushright}/ #1\end{flushright}}

% Separator for equations on a single line.
\newcommand{\eqnsep}{,\quad}

% Quantum Mechanics
\usepackage{braket}

%%%%%%%%%%%%%%%%%%%%%%%%%%%%%%%%%%%%%%%%%%%%%%%%%%%%%%%%%%%%%%%%%%%%%%%%%%%%%%%
%                                  Headings                                   %
%%%%%%%%%%%%%%%%%%%%%%%%%%%%%%%%%%%%%%%%%%%%%%%%%%%%%%%%%%%%%%%%%%%%%%%%%%%%%%%

% This will set fancy headings to the top of the page. The page number will be
% accompanied by the total number of pages. That way, you will know if any page
% is missing.
%
% If you do not want this for your document, you can just use
% ``\pagestyle{plain}``.

\usepackage{scrpage2}

\pagestyle{scrheadings}
\automark{section}
\cfoot{\footnotesize{Seite \thepage\ / \pageref{LastPage}}}
\chead{}
\ihead{}
\ohead{\rightmark}
\setheadsepline{.4pt}

%%%%%%%%%%%%%%%%%%%%%%%%%%%%%%%%%%%%%%%%%%%%%%%%%%%%%%%%%%%%%%%%%%%%%%%%%%%%%%%
%                            Bibliography (BibTeX)                            %
%%%%%%%%%%%%%%%%%%%%%%%%%%%%%%%%%%%%%%%%%%%%%%%%%%%%%%%%%%%%%%%%%%%%%%%%%%%%%%%

\newcommand{\bibliographyfile}{../../../zentrale_BibTeX/Central}
\bibliographystyle{apalike2}

%%%%%%%%%%%%%%%%%%%%%%%%%%%%%%%%%%%%%%%%%%%%%%%%%%%%%%%%%%%%%%%%%%%%%%%%%%%%%%%
%                                Abbreviations                                %
%%%%%%%%%%%%%%%%%%%%%%%%%%%%%%%%%%%%%%%%%%%%%%%%%%%%%%%%%%%%%%%%%%%%%%%%%%%%%%%

\newcommand{\dhabk}{\mbox{d.\,h.}}

%%%%%%%%%%%%%%%%%%%%%%%%%%%%%%%%%%%%%%%%%%%%%%%%%%%%%%%%%%%%%%%%%%%%%%%%%%%%%%%
%                                  Licences                                   %
%%%%%%%%%%%%%%%%%%%%%%%%%%%%%%%%%%%%%%%%%%%%%%%%%%%%%%%%%%%%%%%%%%%%%%%%%%%%%%%

\usepackage{ccicons}

\newcommand{\ccbysadetext}{%
	\begin{small}
		Dieses Werk bzw. Inhalt steht unter einer
		\href{http://creativecommons.org/licenses/by-sa/3.0/deed.de}{%
			Creative Commons Namensnennung - Weitergabe unter gleichen
		Bedingungen 3.0 Unported Lizenz}.
	\end{small}
}

\newcommand{\ccbysadetitle}{%
	Lizenz: \href{http://creativecommons.org/licenses/by-sa/3.0/deed.de}
	{CC-BY-SA 3.0 \ccbysa}
}
`` in your main
% document and start with ``\title`` and ``\begin{document}`` afterwards.

% If you need to add additional packages, I recommend not doing this in this
% file, but in your main document. That way, you can just drop in a new
% ``header.tex`` and get all the new commands without having to merge manually.

% Since this file encorporates a CC-BY-SA fragment, this whole files is
% licensed under the CC-BY-SA license.

\documentclass[11pt, ngerman, fleqn, DIV=15, headinclude, BCOR=2cm]{scrartcl}

\usepackage{graphicx}

% Environment to quote the problem. Currently, this is just a new name for the
% quote environment.
\newenvironment{problem}{\begin{quote}\textsf{\textbf{Aufgabenstellung:}}\quad}{\end{quote}}

\setkomafont{caption}{\sf}
\setkomafont{captionlabel}{\usekomafont{caption}}

%%%%%%%%%%%%%%%%%%%%%%%%%%%%%%%%%%%%%%%%%%%%%%%%%%%%%%%%%%%%%%%%%%%%%%%%%%%%%%%
%                                Locale, date                                 %
%%%%%%%%%%%%%%%%%%%%%%%%%%%%%%%%%%%%%%%%%%%%%%%%%%%%%%%%%%%%%%%%%%%%%%%%%%%%%%%

\usepackage{babel}
\usepackage[iso]{isodate}

%%%%%%%%%%%%%%%%%%%%%%%%%%%%%%%%%%%%%%%%%%%%%%%%%%%%%%%%%%%%%%%%%%%%%%%%%%%%%%%
%                          Margins and other spacing                          %
%%%%%%%%%%%%%%%%%%%%%%%%%%%%%%%%%%%%%%%%%%%%%%%%%%%%%%%%%%%%%%%%%%%%%%%%%%%%%%%

\usepackage[parfill]{parskip}
\usepackage{setspace}
\usepackage[activate]{microtype}

\setlength{\columnsep}{2cm}

%%%%%%%%%%%%%%%%%%%%%%%%%%%%%%%%%%%%%%%%%%%%%%%%%%%%%%%%%%%%%%%%%%%%%%%%%%%%%%%
%                                    Color                                    %
%%%%%%%%%%%%%%%%%%%%%%%%%%%%%%%%%%%%%%%%%%%%%%%%%%%%%%%%%%%%%%%%%%%%%%%%%%%%%%%

\usepackage[usenames, dvipsnames]{xcolor}

\colorlet{darkred}{red!70!black}
\colorlet{darkblue}{blue!70!black}
\colorlet{darkgreen}{green!40!black}

%%%%%%%%%%%%%%%%%%%%%%%%%%%%%%%%%%%%%%%%%%%%%%%%%%%%%%%%%%%%%%%%%%%%%%%%%%%%%%%
%                         Font and font like settings                         %
%%%%%%%%%%%%%%%%%%%%%%%%%%%%%%%%%%%%%%%%%%%%%%%%%%%%%%%%%%%%%%%%%%%%%%%%%%%%%%%

% This replaces all fonts with Bitstream Charter, Bitstream Vera Sans and
% Bitstream Vera Mono. Math will be rendered in Charter.
\usepackage[charter, greekuppercase=italicized]{mathdesign}
\usepackage{beramono}
\usepackage{berasans}

% Bold, sans-serif tensors. This fragment is taken from “egreg” from
% http://tex.stackexchange.com/a/82747/8945 and licensed under `CC-BY-SA
% <https://creativecommons.org/licenses/by-sa/3.0/>`_.
\usepackage{bm}
\DeclareMathAlphabet{\mathsfit}{\encodingdefault}{\sfdefault}{m}{sl}
\SetMathAlphabet{\mathsfit}{bold}{\encodingdefault}{\sfdefault}{bx}{sl}
\newcommand{\tens}[1]{\bm{\mathsfit{#1}}}

% Bold vectors.
\renewcommand{\vec}[1]{\boldsymbol{#1}}

%%%%%%%%%%%%%%%%%%%%%%%%%%%%%%%%%%%%%%%%%%%%%%%%%%%%%%%%%%%%%%%%%%%%%%%%%%%%%%%
%                               Input encoding                                %
%%%%%%%%%%%%%%%%%%%%%%%%%%%%%%%%%%%%%%%%%%%%%%%%%%%%%%%%%%%%%%%%%%%%%%%%%%%%%%%

\usepackage[T1]{fontenc}
\usepackage[utf8]{inputenc}

%%%%%%%%%%%%%%%%%%%%%%%%%%%%%%%%%%%%%%%%%%%%%%%%%%%%%%%%%%%%%%%%%%%%%%%%%%%%%%%
%                         Hyperrefs and PDF metadata                          %
%%%%%%%%%%%%%%%%%%%%%%%%%%%%%%%%%%%%%%%%%%%%%%%%%%%%%%%%%%%%%%%%%%%%%%%%%%%%%%%

\usepackage{hyperref}
\usepackage{lastpage}

% This sets the author in the properties of the PDF as well. If you want to
% change it, just override it with another ``\hypersetup`` call.
\hypersetup{
	breaklinks=false,
	citecolor=darkgreen,
	colorlinks=true,
	linkcolor=darkblue,
	menucolor=black,
	pdfauthor={Martin Ueding},
	urlcolor=darkblue,
}

%%%%%%%%%%%%%%%%%%%%%%%%%%%%%%%%%%%%%%%%%%%%%%%%%%%%%%%%%%%%%%%%%%%%%%%%%%%%%%%
%                               Math Operators                                %
%%%%%%%%%%%%%%%%%%%%%%%%%%%%%%%%%%%%%%%%%%%%%%%%%%%%%%%%%%%%%%%%%%%%%%%%%%%%%%%

% AMS environments like ``align`` and theorems like ``proof``.
\usepackage{amsmath}
\usepackage{amsthm}

% Common math constructs like partial derivatives.
\usepackage{commath}

% Physical units.
\usepackage[output-decimal-marker={,}]{siunitx}

% Since I use mathdesign with italic uppercase greek characters, the Ohm unit will be displayed with an italic Ω by default. Units have to be roman, so this forces it the right way.
\DeclareSIUnit{\ohm}{$\Omegaup$}
\DeclareSIUnit{\division}{DIV}
\DeclareSIUnit{\voltss}{$\mathrm{V_{SS}}$}

% Word like operators.
\DeclareMathOperator{\acosh}{arcosh}
\DeclareMathOperator{\arcosh}{arcosh}
\DeclareMathOperator{\arcsinh}{arsinh}
\DeclareMathOperator{\arsinh}{arsinh}
\DeclareMathOperator{\asinh}{arsinh}
\DeclareMathOperator{\card}{card}
\DeclareMathOperator{\csch}{cshs}
\DeclareMathOperator{\diam}{diam}
\DeclareMathOperator{\sech}{sech}
\renewcommand{\Im}{\mathop{{}\mathrm{Im}}\nolimits}
\renewcommand{\Re}{\mathop{{}\mathrm{Re}}\nolimits}

% Fourier transform.
\DeclareMathOperator{\fourier}{\ensuremath{\mathcal{F}}}

% Roman versions of “e” and “i” to serve as Euler's number and the imaginary
% constant.
\newcommand{\ee}{\eup}
\newcommand{\eup}{\mathrm e}
\newcommand{\ii}{\iup}
\newcommand{\iup}{\mathrm i}

% Symbols for the various mathematical fields (natural numbers, integers,
% rational numbers, real numbers, complex numbers).
\newcommand{\C}{\ensuremath{\mathbb C}}
\newcommand{\N}{\ensuremath{\mathbb N}}
\newcommand{\Q}{\ensuremath{\mathbb Q}}
\newcommand{\R}{\ensuremath{\mathbb R}}
\newcommand{\Z}{\ensuremath{\mathbb Z}}

% Shape like operators.
\DeclareMathOperator{\dalambert}{\Box}
\DeclareMathOperator{\laplace}{\bigtriangleup}
\newcommand{\curl}{\vnabla \times}
\newcommand{\divergence}[1]{\inner{\vnabla}{#1}}
\newcommand{\vnabla}{\vec \nabla}

\newcommand{\half}{\frac 12}

% Unit vector (German „Einheitsvektor“).
\newcommand{\ev}{\hat{\vec e}}

% Scientific notation for large numbers.
\newcommand{\e}[1]{\cdot 10^{#1}}

% Mathematician's notation for the inner (scalar, dot) product.
\newcommand{\bracket}[1]{\left\langle #1 \right\rangle}
\newcommand{\inner}[2]{\bracket{#1, #2}}

% Placeholders.
\newcommand{\emesswert}{\del{\messwert \pm \messwert}}
\newcommand{\fehlt}{\textcolor{darkred}{Hier fehlen noch Inhalte.}}
\newcommand{\messwert}{\textcolor{blue}{\square}}
\newcommand{\punkte}{\phantom{xxxxx}}
\newcommand{\punktevon}[1]{\begin{flushright}/ #1\end{flushright}}

% Separator for equations on a single line.
\newcommand{\eqnsep}{,\quad}

% Quantum Mechanics
\usepackage{braket}

%%%%%%%%%%%%%%%%%%%%%%%%%%%%%%%%%%%%%%%%%%%%%%%%%%%%%%%%%%%%%%%%%%%%%%%%%%%%%%%
%                                  Headings                                   %
%%%%%%%%%%%%%%%%%%%%%%%%%%%%%%%%%%%%%%%%%%%%%%%%%%%%%%%%%%%%%%%%%%%%%%%%%%%%%%%

% This will set fancy headings to the top of the page. The page number will be
% accompanied by the total number of pages. That way, you will know if any page
% is missing.
%
% If you do not want this for your document, you can just use
% ``\pagestyle{plain}``.

\usepackage{scrpage2}

\pagestyle{scrheadings}
\automark{section}
\cfoot{\footnotesize{Seite \thepage\ / \pageref{LastPage}}}
\chead{}
\ihead{}
\ohead{\rightmark}
\setheadsepline{.4pt}

%%%%%%%%%%%%%%%%%%%%%%%%%%%%%%%%%%%%%%%%%%%%%%%%%%%%%%%%%%%%%%%%%%%%%%%%%%%%%%%
%                            Bibliography (BibTeX)                            %
%%%%%%%%%%%%%%%%%%%%%%%%%%%%%%%%%%%%%%%%%%%%%%%%%%%%%%%%%%%%%%%%%%%%%%%%%%%%%%%

\newcommand{\bibliographyfile}{../../../zentrale_BibTeX/Central}
\bibliographystyle{apalike2}

%%%%%%%%%%%%%%%%%%%%%%%%%%%%%%%%%%%%%%%%%%%%%%%%%%%%%%%%%%%%%%%%%%%%%%%%%%%%%%%
%                                Abbreviations                                %
%%%%%%%%%%%%%%%%%%%%%%%%%%%%%%%%%%%%%%%%%%%%%%%%%%%%%%%%%%%%%%%%%%%%%%%%%%%%%%%

\newcommand{\dhabk}{\mbox{d.\,h.}}

%%%%%%%%%%%%%%%%%%%%%%%%%%%%%%%%%%%%%%%%%%%%%%%%%%%%%%%%%%%%%%%%%%%%%%%%%%%%%%%
%                                  Licences                                   %
%%%%%%%%%%%%%%%%%%%%%%%%%%%%%%%%%%%%%%%%%%%%%%%%%%%%%%%%%%%%%%%%%%%%%%%%%%%%%%%

\usepackage{ccicons}

\newcommand{\ccbysadetext}{%
	\begin{small}
		Dieses Werk bzw. Inhalt steht unter einer
		\href{http://creativecommons.org/licenses/by-sa/3.0/deed.de}{%
			Creative Commons Namensnennung - Weitergabe unter gleichen
		Bedingungen 3.0 Unported Lizenz}.
	\end{small}
}

\newcommand{\ccbysadetitle}{%
	Lizenz: \href{http://creativecommons.org/licenses/by-sa/3.0/deed.de}
	{CC-BY-SA 3.0 \ccbysa}
}


\usepackage{placeins}

\ihead{physik313 – Versuch 6}
\ifoot{Lino Lemmer}

\hypersetup{
    pdftitle={Transistorverstärker}
}

\subject{Praktikumsprotokoll}
\title{Transistorverstärker}
\subtitle{physik313 – Versuch 6}
\author{
    Lino Lemmer \footnote{\href{mailto:s6lilemm@uni-bonn.de}{s6lilemm@uni-bonn.de}}
}

%\setcounter{tocdepth}{2}

\newcommand\IB{I_\text{B}}
\newcommand\IC{I_\text{C}}
\newcommand\ID{I_\text{D}}
\newcommand\IE{I_\text{E}}
\newcommand\IS{I_\text{S}}
\newcommand\RC{R_\text{C}}
\newcommand\RE{R_\text{E}}
\newcommand\UBE{U_\text{BE}}
\newcommand\UB{U_\text{B}}
\newcommand\UC{U_\text{C}}
\newcommand\UCE{U_\text{CE}}
\newcommand\UE{U_\text{E}}
\newcommand\UGS{U_\text{GS}}

\usepackage{tocloft}

\newlistof{todo}{lotd}{TODO Liste}

\newcommand{\FIXME}[1]{\printTODO{FIXME: #1}}
\newcommand{\TODO}[1]{\printTODO{TODO: #1}}
\newcommand{\XXX}[1]{\printTODO{XXX: #1}}
\newcommand{\FRAGE}[1]{\printTODO{Rückfrage: #1}}

\newcommand{\printTODO}[1]{
    \par%
    \textcolor{OrangeRed}{\textsf{#1}}%
    \par%
    \refstepcounter{todo}
    \addcontentsline{lotd}{todo}{#1}
}


\begin{document}

\maketitle

Der \LaTeX-Quelltext zu allen Protokollen in diesem Praktikum kann auf
\ref{it:mu} eingesehen werden. Die Quellen für dieses Protokoll können auf
\ref{it:github/alles} eingesehen werden. Die \LaTeX-Datei wird aus
\ref{it:github/template} generiert.

\begin{enumerate}
    \item
        \label{it:mu}
        \url{http://martin-ueding.de/de/university/physik313/}
    \item
        \label{it:github/alles}
        \url{https://github.com/martin-ueding/physik313-3_4/}
    \item
        \label{it:github/template}
        \url{https://github.com/martin-ueding/physik313-3_4/blob/master/Template_4.tex}
\end{enumerate}

\newpage
\tableofcontents

\listoftodo
\newpage

%%%%%%%%%%%%%%%%%%%%%%%%%%%%%%%%%%%%%%%%%%%%%%%%%%%%%%%%%%%%%%%%%%%%%%%%%%%%%%%
%                                 Einleitung                                  %
%%%%%%%%%%%%%%%%%%%%%%%%%%%%%%%%%%%%%%%%%%%%%%%%%%%%%%%%%%%%%%%%%%%%%%%%%%%%%%%

\FloatBarrier
\section{Einleitung}

Nachdem in Versuch~3 verschiedene Transistoreigenschaften untersucht wurden,
vermessen wir in diesem Versuch weitere Schaltungen mit Transistoren.
Insbesondere ist der Transistorverstärker in Emitter- und
Emitterfolgerschaltung zu nennen, bei denen wir die Spannungsverstärkung und
die Abhängigkeit dieser von der Frequenz des Eingangssignals auswerten.

%%%%%%%%%%%%%%%%%%%%%%%%%%%%%%%%%%%%%%%%%%%%%%%%%%%%%%%%%%%%%%%%%%%%%%%%%%%%%%%
%                                  Aufgaben                                   %
%%%%%%%%%%%%%%%%%%%%%%%%%%%%%%%%%%%%%%%%%%%%%%%%%%%%%%%%%%%%%%%%%%%%%%%%%%%%%%%

\FloatBarrier
\section{Aufgaben}

\FloatBarrier
\subsection{Aufgabe O}

\begin{problem}
    Zeigen Sie, dass genauer gilt:
    \begin{equation}
        \label{eq:3_4-14}
        v = - \frac{\beta \RC}{r_\text{BE} + \gamma \RE}
    \end{equation}
\end{problem}

\begin{align*}
    v &= \dod \UC\UB\\
      &= -\frac{\dif\IC\RC}{\dif\UBE + \dif\UE}\\
      &= -\frac{\dif\IC\RC}{\dif\UBE + \dif\IE\RE}\\
      &= -\frac{\od\IC\IB\RC}{\od\UBE\IB+\od\IE\IB\RE}\\
      &= -\frac{\beta\RC}{r_\text{BE}+\gamma\RE}
\end{align*}

\FloatBarrier
\subsection{Aufgabe P}

\begin{problem}
    Beweisen Sie \eqref{eq:3_4-17}.
\end{problem}

Die zitierte Gleichung ist:
\begin{equation}
    \label{eq:3_4-17}
    \frac{\dif  v} v
    = \frac{\dif  v_0}{v_0} \frac{1}{k v_0 + 1}
    = \frac{\dif  v_0}{v_0} \frac{v}{v_0}
\end{equation}

Aus 

\begin{align*}
    \frac 1v &= \frac 1{v_0} + k
    \intertext{folgt:}
    v &= \frac {v_0}{1+kv_0}
    \intertext{Daraus ergibt sich}
    \dod {v}{v_0} &= \frac {1+kv_0-kv_0}{\del{1+kv_0}^2}\\
    &= \frac 1{\del{1+kv_0}^2}\\
    &= \frac {v_0}{1+kv_0} \frac 1{v_0\del{1+kv_0}}\\
    &= \frac v{v_0} \frac 1{1+kv_0}
    \intertext{Hieraus erhält man die gesuchte Gleichung}
    \frac {\dif v}v &= \frac {\dif v_0}{v_0} \frac 1{1+kv_0} = \frac {\dif
v_0}{v_0} \frac v{v_0}
\end{align*}

\FloatBarrier
\subsection{Aufgabe Q}

\begin{problem}
    Erklären sie, wieso die Kapazität $C_\text{CB}$ Einfluss auf die
    Verstärkung hat.
\end{problem}

Bei einer Verstärkung $\beta$ wird durch den Miller-Effekt wird die anfänglich
kleine Kapazität $C_\text{CB}$ um den Faktor $1+\abs\beta$ vergrößert. Die
hierdurch vergrößerte Kapazität wirkt nun als Tiefpass. Die durch diesen
Tiefpass unterdrückten Frequenzen werden zwar weiterhin verstärkt, aber die
Unterdrückung wirkt stärker.

\FloatBarrier
\subsection{Aufgabe R}

\begin{problem}
    Erklären Sie die Funktionsweise der Schaltung in
    Abbildung~\ref{fig:3_4-15}! Wie groß ist die Spannungsänderung im Punkt P
    bei einer Stromänderung $\dif  \IE(T2)$ und welche
    Transistorgröße bestimmt diesen Wert?
\end{problem}

\begin{figure}[htbp]
    \centering
    \includegraphics[width=.6\textwidth]{Anleitung/3_4-15.png}
    \caption{%
        \cite[Abbildung~3/4.15]{physik313-Anleitung}
    }
    \label{fig:3_4-15}
\end{figure}

An $T_2$ liegt eine konstante Spannung an. Durch die Diodeneigenschaft des
Transistors wird dadurch die Spannung am Punkt $P$ auf ca. $\SI 2{\volt} -
\SI{0.6}{\volt} = \SI{1.4}{\volt}$ fixiert, da eine kleine Änderung des Stroms
$\dif\IE$ nahezu keine Änderung der Basis-Emitter-Spannung $\UBE$ in $T_2$
bewirkt.

\FloatBarrier
\subsection{Aufgabe S}

\begin{problem}
    Leiten Sie \eqref{eq:3_4-18} her. Hinweise: Da die Gegenkopplung bei
    Betrachtung im Frequenzraum auf der Addition von Sinusschwingungen beruht
    und wir in diesem Kapitel die Phasen ignoriert haben, ist hier
    $v(f_\text{grenz gk}) = 2v(f=0)$ statt korrekterweise $\sqrt 2 v (f = 0)$.
\end{problem}

Die zitierte Gleichung ist:
\begin{equation}
    \label{eq:3_4-18}
    f_\text{grenz gk} = f_\text{grenz} \frac{v_0}{v(f=0)}
\end{equation}

Die Transitfrequenz $f_\text{T}$ bleibt, trotz der durch die Gegenkopplung
verringerten Verstärkung, die gleiche. Daher gilt
\[f_\text{grenz}v_0 = f_\text{T} = f_\text{grenz gk} v(f=0)\]
Hieraus folgt sofort die gesuchte Gleichung \eqref{eq:3_4-18}.

\FloatBarrier
\subsection{Aufgabe T}

\begin{problem}
    Erläutern Sie die Wirkungsweise der Art der Stabilisierung des
    Basispotentials durch den Widerstand $R$ in Abbildung~\ref{fig:3_4-16}.
    Überlegen Sie dazu, was passiert, wenn das Basispotential aus irgend einem
    (äußeren) Grund „wegläuft“!
\end{problem}

\begin{figure}[htbp]
    \centering
    \includegraphics[width=.6\textwidth]{Anleitung/3_4-16.png}
    \caption{%
        \cite[Abbildung~3/4.16]{physik313-Anleitung}
    }
    \label{fig:3_4-16}
\end{figure}

Da kein Emitterwiderstand in dieser Schaltung vorhanden ist, trennt Basis und
Emitter nur \SI{.6}{\volt}. Der Widerstand zwischen Basis und Masse ist recht
gering ab dieser Spannung, es kann also ein hoher Basisstrom fließen. Dieser
wird jedoch durch $R_\text{in}$ aus $U_\text{in}$ begrenzt.

Durch den Widerstand $R$ wird eine Basisvorspannung angelegt, die jedoch vom
Kollektorstrom abhängt. Im Arbeitspunkt stellt sich folgende Basisspannung ein:
\[
    \UB = U_\text{in} - R_\text{in} I_\text{B in}
    + U_0 - \RC \IC - R I_R
\]

Der Strom $I_\text{B in}$ sollte unabhängig vom Strom $I_R$ sein und somit:
\[
    I_\text{B in} = \frac{U_\text{in} - \SI{.6}\volt}{R_\text{in}}
\]

Eingesetzt bleibt von $\UB$ nur noch folgendes übrig:
\[
    \UB = \SI{.6}\volt + U_0 - \RC \IC - R I_R
\]

Das liegt daran, dass $U_\text{in}$ einen so hohen Basistrom erzeugen würde,
bis der Widerstand $R_\text{in}$ die meiste Spannung wieder verbrannt hat. Die
Basisvorspannung muss also über $R$ laufen. Da über $R_\text{in}$ nur das
kleine Wechselstromsignal kommen soll, passt dies auch.

Je nach Bemessung von $R$ und $\RC$ wird sich ein gewisser Arbeitspunkt mit
einer gewissen Basisspannung $\UB$ einstellen. Darauf hin wird ein Basisstrom
$\IB$ fließen, der jedoch auf einen recht kleinen Widerstand trifft, solange
$\UB \gtrsim \SI{.6}\volt$ gilt. Würde $\UB$ konstant bleiben, käme es hier zu
einem beliebigen Anwachsen von $\IB$.

Jedoch bedeutet ein höherer Basisstrom, der durch $R$ fließen muss, dass mehr
Spannung über diesem abfällt und somit die Basisvorspannung gesenkt wird.
Außerdem schaltet der Transistor einen höhreren Kollektorstrom $\IC$ frei,
womit auch wieder mehr Strom über $\RC$ abfällt, das Basispotential rückt
wieder näher zur Masse, es fließt weniger Basisstrom.

Auf diese Weise wird ein unbeschränktes Anwachsen von $\IB$, und somit die
Zerstörung des Transistors, verhindert werden.

\FloatBarrier
\subsection{Aufgabe U}

\begin{problem}
    An welche Stelle der Schaltung [Abbildung~\ref{fig:3_4-16}] würden Sie den
    Kondensator setzen [, um die Rückkopplung wechselspannungsmäßig
    aufzuheben]? (Tip: Man kann einen Widerstand teilen!)
\end{problem}

Um die Rückkopplung für Wechselspannungen aufzuheben, muss die Leitfähigkeit
von $R$ für Wechselspannungen herabgesetzt werden. Dies wird ermöglicht, indem
man den Widerstand $R$ aufteilt und in die Mitte einen Kondensator zur Masse
einsetzt. Der so entstandene Tiefpass bewirkt eine Unterdrückung hoher
Frequenzen und wirkt in beide Richtungen.

%%%%%%%%%%%%%%%%%%%%%%%%%%%%%%%%%%%%%%%%%%%%%%%%%%%%%%%%%%%%%%%%%%%%%%%%%%%%%%%
%                    Durchführung: Transistorverstärker                     %
%%%%%%%%%%%%%%%%%%%%%%%%%%%%%%%%%%%%%%%%%%%%%%%%%%%%%%%%%%%%%%%%%%%%%%%%%%%%%%%

\FloatBarrier
\section{Durchführung}

\FloatBarrier
\subsection{Fortsetung Emitterfolger}

\subsubsection{Spannungsverstärkung des Emitterfolgers}

Auf Schaltbrett~1 (Abbildung~\ref{fig:3-4}) wird ein Emitterfolger mit externem
Offset aufgebaut. Dazu wird der Funktionsgenerator auf Sinussignal mit
$U_\text{SS} = \SI{0.5}{\voltss}$ und Offset $\UB = \SI2{\volt}$ eingestellt.
An die Stellen der waagerechten Doppelpfeile kommen einfache Kabelbrücken.
Kanal~2 des Oszilloskops wird mit dem unteren Anschluss verbunden.

\begin{figure}[htbp]
    \centering
    \includegraphics[width=\textwidth]{Anleitung/3-4.png}
    \caption{%
        Schaltbrett~1 \cite[Abbildung~3.4]{physik313-Anleitung}
    }
    \label{fig:3-4}
\end{figure}

Nun wird zunächst der Emitterwiderstand $\RE$ konstant gelassen und die
Spannungsverstärkung für drei verschiedene Kollektorwiderstände $\RC$ gemessen,
danach umgekehrt.

Die Messung ist in Tabelle~\ref{tab:411}.

\begin{table}[htbp]
    \centering
    \begin{tabular}{SSSS|S}
        {$\RE / \si{\ohm}$} &
        {$\RC / \si{\ohm}$} &
        {$\dif\UB / \si{\division}$} &
        {$\dif\UE / \si{\division}$} &
        {$v = \dif\UE / \dif\UB$} \\
        \hline
        %< for re, rc, dub, due, v in table411: >%
        << re >> & << rc >> & << dub >> & << due >> & << v >> \\
        %< endfor >%
    \end{tabular}
    \caption{%
        Abhängigkeit der Spannungsverstärkung von $\RE$ und $\RC$
    }
\label{tab:411}
\end{table}

Die Erwartung, dass sich die Verstärkung nicht nennenswert verändert, da sie in
erster Näherung 
\[
    v = \frac{\gamma\RE}{r_\text{BE}+\gamma\RE} \approx 1
\]
unabhängig sein soll von $\RE$ und $\RC$ wurde nur für kleine $\RC$ und große
$\RE$ bestätigt. Die Werte von $v$ schwanken insgesamt aber deutlich. Bei sehr
kleinem $\RE$ wurde die Verstärkung groß, während bei sehr großem $\RC$ die
Verstärkung kleiner wurde.

\subsubsection{Emitterfolger als Impedanzwandler}

Nun wird ein invertierender Transistorverstärker aufgebaut, indem in
Abbildung~\ref{fig:3-4} $\RC = \SI{22}{\kilo\ohm}$ und $\RE = \SI1{\kilo\ohm}$
gesetzt werden. Als Eingangssignal wird ein Sinussignal mit $U_\text{SS} = \SI
{0.5}{\voltss}$, $\UB = \SI1{\volt}$ und $f = \SI{800}{\hertz}$ gewählt.

Nun wird an den Kollektor ein Lautsprecher mit $R \approx \SI{300}{\ohm}$
angeschlossen. 

Wir hören fast nichts, da die Emitterschaltung nur die Spannung verstärkt,
jedoch nicht die Leistung.

Mit dem Schaltbrett~2 (Abbildung~\ref{fig:3-5}) wird ein Emitterfolger zwischen
Transistorverstärker und Lautsprecher eingebaut. Dazu wird die
Gleichspannungsversorgung von Brett~1 übernommen und der Lautsprecher wird an
den Emitter des npn-Bipolar-Transistors angeschlossen.

\begin{figure}
    \centering
    \includegraphics[width=\textwidth]{Anleitung/3-5.png}
    \caption{%
        Schaltbrett~2 \cite[Abbildung~3.5]{physik313-Anleitung}
    }
    \label{fig:3-5}
\end{figure}

Jetzt ist ein Ton zu hören. Die Kollektorschaltung lässt die Spannung gleich,
diese wurde allerdings schon verstärkt. Jetzt wird der Strom verstärkt. Nach
beiden Stufen wurde somit die Leistung verstärkt, diese reicht jetzt für den
Lautsprecher aus.

\FloatBarrier
\subsection{Invertierender Transistorverstärker (Emitterschaltung)}

\FloatBarrier
\subsubsection{Phasenbeziehung zwischen Ein- und Ausgang}

Auf Schaltbrett~1 (Abbildung~\ref{fig:3-4}) wird nun eine Emitterschaltung
aufgebaut, indem Kanal~2 des Oszilloskops an den oberen Ausgang geschlossen
wird. Für Kollektor- und Emitterwiderstand soll gelten: $\RC = \RE =
\SI{390}{\ohm}$.

Das Basispotential wird mit dem Spannungsteiler (senkrechter Doppelpfeil
überbrückt) so eingestellt, dass $\UB = \SI{1.5}{\volt}$ ist. Eine der
waagerechten Kabelbrücken aus dem vorherigen Versuch muss durch einen möglichst
großen Kondensator ersetzt werden, damit der Gleichspannungsanteil aus dem
Signal des Generators gefiltert wird. Letzterer soll ein
\SI{0.5}{\voltss}-Sinussignal ausgeben.

Die Phasenbeziehung zwischen Eingangs- und Ausgangssignal ist hierbei $\pi$.
Dazu haben wir noch ein Foto, siehe Abbildung~\ref{fig:799}.

\begin{figure}[htbp]
	\centering
	\includegraphics[width=.45\linewidth]{Oszi_Foto/4-799.jpg}
	\caption{%
		Verstärkung \SI{50}{\milli\volt\per\division},
		Zeitbasis \SI{.2}{\milli\second\per\division},
		AC~Kopplung
	}
	\label{fig:799}
\end{figure}

\FloatBarrier
\subsubsection{Spannungsverstärkung des invertierenden Verstärkers}

Wir vermessen die Abhängigkeit der Spannungsverstärkung für verschiedene $\RC$
und $\RE$, siehe Tabelle~\ref{tab:412}. Zum Vergleich mit der Näherungsformel
ist eine Spalte mit der genäherten Verstärkung eingefügt.

\begin{table}[htbp]
    \centering
    \begin{tabular}{SSSS|SS}
        {$\RE / \si\ohm$} &
        {$\RC / \si\ohm$} &
        {$\dif\UB / \si\division$} &
        {$\dif\UE / \si\division$} &
        {$v$} &
        {$- \RC / \RE$} \\
        \hline
        %< for re, rc, dub, due, v, rcre in table412: >%
        << re >> & << rc >> & << dub >> & << due >> & << v >> & << rcre >> \\
        %< endfor >%
    \end{tabular}
    \caption{%
        Abhängigkeit der Spannungsverstärkung. Vor der vertikalen Linie sind
        Messwerte, dahinter berechnete Werte.
    }
    \label{tab:412}
\end{table}

Die Näherungsformel ist für große $\RE$ ziemlich gut, für große $\RC$ ziemlich
schlecht. Wobei dies auch an starken Schwankungen und damit verbundenen
Ableseschwierigkeiten im Bereich der großen $\RC$ liegen könnte.

\FloatBarrier
\subsubsection{Bestimmung des Transistoreingangswiderstands}

Wir bestimmen die Spannungsverstärkung für den Spezialfall:
\[
    \RE = \SI0\ohm
    \eqnsep
    \RC = \SI{390}\ohm
    \eqnsep
    \dif\UB = \SI{.05}\voltss
    \eqnsep
    U = \SI{.7}V
\]

Dabei haben wir die Leerlaufverstärkung erhalten:
\[
    \dif\UB = \SI{<< ew_dUB >>}\division
    \eqnsep
    \dif\UC = \SI{<< ew_dUC >>}\division
    \eqnsep
    v_0 = \dod\UC\UB = \num{<< ew_v0 >>}
\]

\begin{figure}[htbp]
    \centering
    \includegraphics[width=\textwidth]{vRE.pdf}
    \caption{%
        Erstes Diagramm zur Auswertung. Dabei sind hier die ersten Einträge
        aus Tabelle~\ref{tab:412}.
    }
    \label{fig:vRE}
\end{figure}

Als Fitparameter mit einer linearen Funktion für die Daten aus
Abbildung~\ref{fig:vRE} erhalten wir eine Steigung von \num{<< steigung >>} und
einen Achsenabschnitt von \num{<< achsenabschnitt >>}. Daraus lässt sich eine
Leerlaufverstärkung von \num{<< v0_abschnitt >>} bestimmen.

\begin{figure}[htbp]
    \centering
    \includegraphics[width=\textwidth]{vRC.pdf}
    \caption{%
        Zweites Diagramm zur Auswertung. Dabei sind hier die letzten Einträge
        aus Tabelle~\ref{tab:412}.
    }
    \label{fig:vRC}
\end{figure}

Die Daten in Abbildung~\ref{fig:vRC} sind, wie schon aus Tabelle~\ref{tab:412}
zu erwarten ist, nicht sinnvoll brauchbar.

Wir gehen davon aus, dass nur die ersten beiden Punkte für eine Auswertung
relevant sind, dies reicht aber nicht aus um einen fundierten Fit zu erstellen.

Da der Versuchsteil, mit dem hier verglichen werden soll bei uns ebenfalls
keine sinnvollen Ergebnisse geliefert hat, sparen wir uns hier sinnlose
Rechnerei.

\FloatBarrier
\subsection{Wechselstrommäßige Aufhebung der Gegenkopplung}

Wir betrachten den Spielraum der Basisvorspannung bei der vorherigen Schaltung.

Bei $\RE = 0$ haben wir bei einer Basisvorspannung im Bereich
\SIrange{.6}{.8}{\volt} noch eine Sinuskurve im Ausgang sehen können. Bei
\SI{390}{\ohm} für $\RE$ und $\RC$ haben wir als minimale Basisvorspannung
\SI{.8}{\volt} erhalten, ein Maximum konnten wir jedoch nicht finden, da das
Potentiometer am Anschlag war.

Nun schalten wir einen Kondensator parallel zu $\RE$, die Gegenkopplung für
Wechselströme aufzuheben. Dies bedeutet, dass der Kondensator möglichst klein
sein muss, um für große Frequenzen einen kleinen Widerstand darzustellen ($X =
1/(\ii \omega C)$). Wir wählen \SI{3300}{\pico\farad}. Für ein Frequenzspektrum
über fast alle möglichen Frequenzen des Generator nehmen wir die Verstärkung
auf. Diese Daten sind in Tabelle~\ref{tab:413}.

\begin{table}[htbp]
    \centering
    \begin{tabular}{SSS|S}
        {$f / \si\hertz$} &
        {$\dif\UB / \si\division$} &
        {$\dif\UE / \si\division$} &
        {$|v|$} \\
        \hline
        %< for f, dub, due, v in table413: >%
        << f >> & << dub >> & << due >> & << v >> \\
        %< endfor >%
    \end{tabular}
    \caption{%
        Messwerte für die Wechselspannungsverstärkung mit einer parallelen
        Kapazität.
    }
    \label{tab:413}
\end{table}

Die hohe Verstärkung von \num{<< ew_v0 >>} haben wir nicht wieder erreichen
können. Jedenfalls ist ab einer Frequenz von ungefähr \SI{500}{\kilo\hertz} die
Verstärkung größer als 1, so dass die wechselstrommäßige Aufhebung
funktioniert. Dass die letzte Verstärkung wieder kleiner wird, liegt daran,
dass hier die Grenze des Verstärkers erreicht ist, siehe nächste Aufgabe.

\FloatBarrier
\subsection{Frequenzverhalten und Kaskodenschaltung}

Wir bauen eine Verstärkerschaltung auf dem Schaltbrett~2 auf. Dazu wählen wir
$\RE = \SI{100}\ohm$ und, um eine Verstärkung von 10 zu erreichen, $\RC =
\SI1{\kilo\ohm}$. Das Eingangssignal ist ein Sinus mit
$\SI{100}{\milli\voltss}$, den Arbeitspunkt stellen wir mit dem Offset am
Generator ein.

Da Schaltbrett~2, im Gegensatz zu Schalbrett~1 an dieser Stelle über keinen
\SI{10}{\kilo\ohm}-Einganswiderstand verfügt, benutzen wir noch einen solchen
Widerstand mit BNC-Anschluss.

Ab einer Frequenz von \SI{100}{\hertz} untersuchen wir die
Spannungsabhängigkeit dieser Schaltung. Die Daten sind in
Tabelle~\ref{tab:414normal}. Dabei scheint das Netzgerät ein Sägezahnmuster mit
einer recht hohen Frequenz zu haben. Dadurch wurde die Betriebsspannung leicht
varriiert, wodurch die Verstärkung auch variiert ist, siehe
Abbildung~\ref{fig:800}. Dieser Effekt ist auch in anderen Gruppen aufgetaucht.
Für die Messung haben wir den Mittelwert des Musters genommen.

\begin{table}[htbp]
    \centering
    \begin{tabular}{SSS|S}
        {$f / \si\hertz$} &
        {$\dif\UB / \si\division$} &
        {$\dif\UE / \si\division$} &
        {$|v|$} \\
        \hline
        %< for f, dub, due, v in table414normal: >%
        << f >> & << dub >> & << due >> & << v >> \\
        %< endfor >%
    \end{tabular}
    \caption{%
        Messwerte für die Wechselspannungsverstärkung.
    }
    \label{tab:414normal}
\end{table}

\begin{figure}[htbp]
	\centering
	\begin{minipage}{.3\linewidth}
		\includegraphics[width=\linewidth]{Oszi_Foto/4-800.jpg}
	\end{minipage}
	\hfill
	\begin{minipage}{.3\linewidth}
		\includegraphics[width=\linewidth]{Oszi_Foto/4-801.jpg}
	\end{minipage}
	\hfill
	\begin{minipage}{.3\linewidth}
		\includegraphics[width=\linewidth]{Oszi_Foto/4-802.jpg}
	\end{minipage}
	\caption{%
		Links:~Verstärkungen~\SI{50}{\milli\volt\per\division}~und~\SI{.5}{\milli\volt\per\division},~Zeitbasis~\SI{.5}{\milli\second\per\division}.
		Mitte:~Verstärkungen~\SI{20}{\milli\volt\per\division}~und~\SI{.2}{\milli\volt\per\division},~Zeitbasis~\SI{50}{\micro\second\per\division}.
		Rechts:~Verstärkungen~\SI{10}{\milli\volt\per\division}~und~\SI{.1}{\milli\volt\per\division},~Zeitbasis~\SI{.1}{\micro\second\per\division}.
	}
	\label{fig:800}
\end{figure}

Anschließend erweitern wir die Schaltung zu einer Kaskodenschaltung. Die
Messung wiederholen wir. Die entsprechenden Daten sind in
Tabelle~\ref{tab:414normal}.

\begin{table}[htbp]
    \centering
    \begin{tabular}{SSS|S}
        {$f / \si\hertz$} &
        {$\dif\UB / \si\division$} &
        {$\dif\UE / \si\division$} &
        {$|v|$} \\
        \hline
        %< for f, dub, due, v in table414Kaskode: >%
        << f >> & << dub >> & << due >> & << v >> \\
        %< endfor >%
    \end{tabular}
    \caption{%
        Messwerte für die Wechselspannungsverstärkung der Kaskodenschaltung
    }
    \label{tab:414Kaskode}
\end{table}

Die Daten aus beiden Messungen haben wir in Abbildung~\ref{fig:414plot}
geplottet.

\begin{figure}
    \centering
    \includegraphics[width=\linewidth]{4-1-4-Plot.pdf}
    \caption{%
        Bode-Plot für die Verstärkerschaltungen. In rot die Werte für den
        normalen Verstärker, in blau mit Kaskode.
    }
    \label{fig:414plot}
\end{figure}

\TODO{Bandbreite und Transitfrequenz ermitteln. Dafür geraden in Plot einfügen}

\FloatBarrier
\subsection{Verstärker mit Spannungsgegenkopplung}

Wir bauen nun die Schaltung aus Abbildung~\ref{fig:4-1} auf. Wir stellen am
Signalgenerator eine Frequenz von $f = \SI 1{\kilo\hertz}$ ein.

\begin{figure}[htbp]
    \centering
    \includegraphics[width=.6\textwidth]{Anleitung/4-1.png}
    \caption{
        \cite[Abbildung~4.1]{physik313-Anleitung}
    }
    \label{fig:4-1}
\end{figure}

Nun wird zunächst die Spannungsverstärkung vermessen, indem wir das
Eingangssignal vor dem Eingangswiderstand $R_\text{in}$ abnehmen. Dabei messen
wir einmal mit $R = \SI {100}{\kilo\ohm}$ und einmal mit $R = \SI
{33.9}{\kilo\ohm}$, da dieser den \SI
{33}{\kilo\ohm}, die in der Anleitung gefordert wurden, am nächsten kam.

Unsere Messung befindet sich in Tabelle~\ref{tab:415}. Es ist deutlich zu sehen
wie die Verstärkung anwächst, wenn man hinter dem Eingangswiderstand misst.
Dies ist durch den Spannungsabfall über $R_\text{in}$ einfach zu erklären.

Ebenfalls zu erkennen ist, dass die Verstärkung größer wird, je größer der
Widerstand $R$ ist.

\begin{table}
    \centering
    \begin {tabular}{SSSSS|S|r}
        {$R / \si{\kilo\ohm}$} & 
        {$\dif U_\text{in} / \si\division$} &
        {\si{\milli\volt\per\division}}&
        {$\dif U_\text{out} / \si\division$} &
        {\si{\milli\volt\per\division}} &
        {$\abs v$} &
        vor/hinter $R_\text{in}$\\
        \hline
        100 & -3.5 & 100 & 2.9 & 1\si{\kilo} & 8.286 & vor \\
        100 & 2.6 & 20 & 2.6 & 1\si{\kilo} & 50 & hinter \\
        \hline
        33.9 & -3.5 & 100 & 2.25 & 500 & 3.214 & vor \\
        33.9 & 2.8 & 20 & 2.2 & 500 & 19.643 & hinter \\
    \end{tabular}
    \caption{%
        Spannungsverstärkung bei Spannungsgegenkopplung
    }
    \label{tab:415}
\end{table}



%%%%%%%%%%%%%%%%%%%%%%%%%%%%%%%%%%%%%%%%%%%%%%%%%%%%%%%%%%%%%%%%%%%%%%%%%%%%%%%
%                                  Literatur                                  %
%%%%%%%%%%%%%%%%%%%%%%%%%%%%%%%%%%%%%%%%%%%%%%%%%%%%%%%%%%%%%%%%%%%%%%%%%%%%%%%

\FloatBarrier
\IfFileExists{\bibliographyfile}{
    \bibliography{\bibliographyfile}
}{}

\end{document}

% vim: spell spelllang=de tw=79
